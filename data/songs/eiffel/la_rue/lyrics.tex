\begin{multicols}{2}

	\couplet{intro}{}

	\couplet{couplet 1 \songbookcomment{10 mesures}}{\\
		\includegraphics{mps/couplet-0} \\
		À chacun de nos souffles, au moindre murmure des bas fonds \\
		C'est dans l'air comme un chant qui s'étrangle \\
		Que d'un pavé de fortune contre le tintamarre du pognon \\
		À tout moment la rue peut aussi dire non \\
		À tout moment la rue peut aussi dire non
	}

	\couplet{couplet 2}{    \\
		C'est un pincement de lèvres et la peur qui perle d'un front \\
		La faune et la flore à cran en haillons \\
		Et l'éclat de nos palpitants dans l'ombre du marteau pilon \\
		À tout moment la rue peut aussi dire non \\
		À tout moment la rue peut aussi dire non }

	\refrain{refrain 1}{ \\
		Non comme un oui, Aux arbres chevelus \\
		À tout ce qui nous lie, Quand la nuit remue \\
		Aux astres et aux Déesses \\
		Qui peuplent nos rêves et quand le peuple cr\`eve \\
		À tout moment la rue peut aussi dire \\
	}

	\couplet{couplet 3}{    \\
		Et si quelques points noirs en cols blancs poivrent nos cieux \\
		D'ondes occultes en tubes longs et creux \\
		À bien compter, le monde est x fois plus nombreux \\
		Que ces trois cent familles qui sur la rue ont pignon \\
		À tout moment elle peut aussi dire non \\
		\\
		Comme un oui, Aux arbres chevelus \\
		À tout ce qui nous lie, Quand la nuit remue \\
		Aux astres et aux Déesses \\
		Qui peuplent nos rêves et quand le peuple rêve il aime \\
		Disposer de lui-même, Disposer de lui-même \\
		\\
		Non comme un oui, Aux arbres chevelus \\
		À toutes ces nuits qui nous lient \\
		Et même si elles ont trop bu \\
		C'est aux astres et aux Déesses \\
		Qui peuplent nos rêves et quand le peuple crève \\
		À tout moment la rue peut aussi dire \\
		À chacun de nos souffles, au moindre murmure des bas fonds \\
		C'est dans l'air comme un chant qui s'étrangle \\
		Que d'un pavé de fortune contre le tintamarre du pognon \\
		\\
		À tout moment la rue peut aussi dire non ( x... )\\
		\\
	}

\end{multicols}
