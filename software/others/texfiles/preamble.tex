%\usepackage[margin=0.5cm]{geometry}
\usepackage[left=1cm,right=1cm,top=1cm,bottom=2cm]{geometry}
\usepackage[utf8]{inputenc}
%\usepackage[light, largesmallcaps]{kpfonts}
\usepackage{mwe}
\usepackage{multicol}
\usepackage{pgfkeys}

%\usepackage{geometry}
\usepackage{lipsum}
\usepackage{microtype}
\usepackage{grid-system}
\usepackage{babel}
\usepackage{listings}
\usepackage{graphicx}
\usepackage{verse}
\usepackage{setspace}

\usepackage{aurical}
\usepackage[T1]{fontenc}
%\usepackage{css-colors}
\usepackage{latexcolors}
%\usepackage[table,svgnames]{xcolor}
\usepackage{tabularray}
\usepackage{nicematrix}
%\usepackage{titlesec}
%\usepackage{color}
%\usepackage{longtable}
%\usepackage{multirow}
%\usepackage[full]{leadsheets}

% grilles
\newlength\gridheight\setlength\gridheight{215.0mm}
\newlength\gridwidth\setlength\gridwidth{180.0mm}
\def\mygridbox#1{\parbox[c][\gridheight][c]{\gridwidth}{\center #1}}
\def\mytitlebox#1{\parbox[c][{40.0mm}][c]{\gridwidth}{\center #1}}
\setlength{\tabcolsep}{0.0mm}

% we want only the chapters in the toc, for books. No toc for rsongs
\setcounter{tocdepth}{1}

\usepackage{fancyhdr}
\usepackage{lastpage}
\pagestyle{fancy}
\fancyhf{}
%\usepackage{tabularx}
\usepackage[most]{tcolorbox}

\usepackage{booktabs}
\usepackage{array}
\usepackage{paracol}

\usepackage{hyperref}
\hypersetup{
	colorlinks,
	citecolor=black,
	filecolor=black,
	linkcolor=black,
	urlcolor=black
}

%\setlength{\headheight}{10pt}
\usepackage[normalem]{ulem}
\renewcommand{\headrulewidth}{0pt}
\renewcommand{\footrulewidth}{2pt}

%\lhead{\includegraphics[width=1cm]{example-image-a}}
\rhead{}

\rfoot{\thepage/\pageref{LastPage}}
%\cfoot{\today}
%\cfoot{derni\`ere modif le \songlastupdate, g\'en\'er\'e le \songtoday }
\cfoot{derni\`ere modif le \songlastupdate }

\definecolor{bg}{HTML}{ADFF2F}
%\definecolor{bgred}{\color{Seashell3}}

\usepackage{fontspec}
\setmainfont{Garamond Libre}

% pour les lignes pointillées dans les tableaux
\usepackage{arydshln}

\newcommand{\lolocomment}[1]{{
			\fontsize{8pt}{8pt}\selectfont
			\textcolor{red}{#1}}
}

\newcommand{\basecouplet}[3]{
	\colorbox{#1}{
		\fontsize{18pt}{18pt}\selectfont
		#2
	}
	{
		{
				\fontsize{16pt}{16pt}\selectfont
				#3
			}
	}
}

\newcommand{\basetitle}[2]{
	\colorbox{#1}{
		\fontsize{12pt}{12pt}\selectfont
		#2
	}
}

\newcommand{\xxmakesongtitle}[3] {
	\begin{center}
		{
			\Fontskrivan\bfseries\slshape
			\fontsize{60pt}{50pt}\selectfont
			\color{blue}
			#1
		} \\


		{
		\Fontskrivan\bfseries\slshape
		\fontsize{20pt}{10pt}\selectfont
		\color{orange}
		#2
		}
		{
		%\Fontskrivan\bfseries\slshape
		\fontsize{10pt}{5pt}\selectfont
		\color{blue}
		#3 BPM
		}\\
	\end{center}

}


\newcolumntype{C}{>{\centering\arraybackslash}m{1.5cm}}
%\newcolumntype{C}{>{\centering\arraybackslash}m{3.5cm}}
\newcolumntype{R}{r}
\newcolumntype{D}{>{\centering\arraybackslash}m{1cm}}
%\newcolumntype{C}{>{\centering}p{0.28cm}}

\newcommand{\beforegrille}{
	%\Fontskrivan\bfseries\slshape
	%\fontsize{55pt}{52pt}
	\fontsize{10pt}{8pt}\selectfont
	\renewcommand{\arraystretch}{2}

}

\newcommand{\beforelyrics}{
	%\Fontskrivan\bfseries\slshape
	\Fontskrivan
	\fontsize{55pt}{52pt}
	%   \fontsize{10pt}{8pt}\selectfont
	\renewcommand{\arraystretch}{2}

}


\usepackage[thinlines]{easytable}


%\usepackage{paracol}

\newfontfamily\songbookfont[%%basic weight: 100, "bold" weight: 70
	Extension      = .ttf,
	ItalicFont     = songbook,
	BoldFont       = songbook,
	BoldItalicFont = songbook]
{songbook}



\newfontfamily\songbookfontflat[%%basic weight: 50, "bold" weight: 70\overline{}
	Extension      = .ttf,
	ItalicFont     = songbook_flat,
	BoldFont       = songbook_flat,
	BoldItalicFont = songbook_flat]
{songbook_flat}


\newfontfamily\songbookfontsharp[%%basic weight: 50, "bold" weight: 70
	Extension      = .ttf,
	ItalicFont     = songbook_sharp,
	BoldFont       = songbook_sharp,
	BoldItalicFont = songbook_sharp]
{songbook_sharp}


% pour les livres
\newcommand{\fakesection}[1]{%
	\par\refstepcounter{section}% Increase section counter
	\sectionmark{#1}% Add section mark (header)
	\addcontentsline{toc}{section}{\protect\numberline{\thesection}#1}% Add section to ToC
	% Add more content here, if needed.
}


%\usepackage{color, colortbl}
%\usepackage[table]{xcolor}
\definecolor{Gray}{gray}{0.9}
\definecolor{LightCyan}{rgb}{0.88,1,1}
\definecolor{Row1}{rgb}{0.77,9,9}
\definecolor{Row2}{rgb}{0.9,0.77,9}

%LightGoldenrodYellow 	  	FA 	FA 	D2 	250 	250 	210
\definecolor{LightGoldenrodYellow}{RGB}{250,250,210}
\newcommand{\rowcolora}{\rowcolor{Thistle1}}
%\newcommand{\rowcolorb}{\rowcolor{Seashell3}}
\newcommand{\rowcolorb}{\rowcolor{Coral1}}


\newcounter{bar}

% bar counter
\newcommand{\loloinitcounter}[1]{
	\setcounter{bar}{#1}
}

\newcommand{\loloshowcounter}[1]{%
	\fontsize{16pt}{6pt}
	\cellcolor{white}$_{\arabic{bar}}$
	\addtocounter{bar}{#1}
}



\newcommand{\loloshowarrowcounter}[1]{%
	\fontsize{16pt}{6pt}
	\cellcolor{white}$_{\arabic{bar} \rightarrow \addtocounter{bar}{#1} \addtocounter{bar}{-1} \arabic{bar}}$
	\addtocounter{bar}{1}
}

\newcommand{\loloprintbpm}[3]{%
	% min sec bars

	\def\x{#1}
	\def\y{#2}
	\def\z{#3}
	% $\x \div \y =$
	\newcount\a\a=\number\x
	\newcount\b\b=\number\y
	\multiply\a by 60
	\advance\a \y
	\def\nseconds{\the\a}
	%number of seconds : $\nseconds$

	\newcount\c\c=\number\z
	\multiply\c by 4
	\def\nbbars{\the\c}
	%number of beats : $\nbbars$

	\multiply\c by 60
	\divide\c by \nseconds
	\def\bpm{\the\c}
	%number of beats per minute : $\bpm$

	durée : #1' #2'' ; #3 mesures ; tempo = $\bpm$

	% #1' #2'/$\bpm$

}

\newcommand{\makefooter}[2]{
	\lfoot{#1/#2}
	%/\loloprintbpm{#3}{#4}{#5}}
}

\newcommand{\lolomakerowanc}[1]{
	& \multicolumn{4}{l}{#1}
}


\newcommand{\lolomakerowbnc}[1]{
	& \multicolumn{1}{l}{}         & \multicolumn{3}{l}{#1}
}

\newcommand{\lolomakerowcnc}[1]{
	& \multicolumn{1}{l}{}
	& \multicolumn{1}{l}{}         & \multicolumn{2}{l}{#1}
}

\newcommand{\lolomakerowdnc}[1]{
	& \multicolumn{1}{l}{}
	& \multicolumn{1}{l}{}
	& \multicolumn{1}{l}{}         & \multicolumn{1}{l}{#1}
}

\newcommand{\lolomakerowa}[2]{
	\loloshowarrowcounter{#1}
	& \multicolumn{4}{l}{#2}
}


\newcommand{\lolomakerowb}[2]{
	\loloshowarrowcounter{#1}
	& \multicolumn{1}{l}{}         & \multicolumn{3}{l}{#2}
}

\newcommand{\lolomakerowc}[2]{
	\loloshowarrowcounter{#1}
	& \multicolumn{1}{l}{}
	& \multicolumn{1}{l}{}         & \multicolumn{2}{l}{#2}
}

\newcommand{\lolomakerowd}[2]{
	\loloshowarrowcounter{#1}
	& \multicolumn{1}{l}{}
	& \multicolumn{1}{l}{}
	& \multicolumn{1}{l}{}         & \multicolumn{1}{l}{#2}
}


\newcommand\lolotwocolumns[2]{
	\begin{minipage}{0.50\linewidth}
		\vspace{0pt}
		#1
	\end{minipage}
	\begin{minipage}{0.40\linewidth}
		\vspace{0pt}
		#2
	\end{minipage}
}

\newcommand\lolothreecolumns[3]{
	\begin{minipage}{0.40\linewidth}
		\vspace{0pt}
		#1
	\end{minipage}
	\begin{minipage}{0.10\linewidth}
		\vspace{0pt}
		#2
	\end{minipage}
	\begin{minipage}{0.50\linewidth}
		\vspace{0pt}
		#3
	\end{minipage}
}

\newcommand\lolohr{
	\begin{center}
		\line(1,0){450}
	\end{center}
}

\newcommand\lolovspace{\vspace{0.5cm plus 0.5ex}}
\newcommand\lolohspace{\hspace{5cm}}


\newcommand{\setvalue}[1]{\pgfkeys{/variables/#1}}
\newcommand{\getvalue}[1]{\pgfkeysvalueof{/variables/#1}}
\newcommand{\declare}[1]{%
	\pgfkeys{
		/variables/#1.is family,
		/variables/#1.unknown/.style = {\pgfkeyscurrentpath/\pgfkeyscurrentname/.initial = ##1}
	}%
}

\declare{}

\newcommand{\songbooksongstruct}{}
\newcommand{\xxxlyrics}{}

\newcommand{\songly}[1]{ %
	\subimport{#1.output}{#1}
}
